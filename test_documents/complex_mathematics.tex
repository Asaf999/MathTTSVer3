\documentclass{article}
\usepackage{amsmath}
\usepackage{amssymb}
\usepackage{amsthm}

\newtheorem{theorem}{Theorem}
\newtheorem{lemma}[theorem]{Lemma}
\newtheorem{corollary}[theorem]{Corollary}

\title{Advanced Mathematical Concepts: A Comprehensive Test}
\author{Phase 3.5 Testing System}
\date{\today}

\begin{document}

\maketitle

\section{Introduction}
This document contains challenging mathematical expressions to test the naturalness of our LaTeX to speech conversion system.

\section{Complex Analysis and Euler's Identity}

\begin{theorem}[Euler's Formula]
For any real number $\theta$, we have:
\begin{equation}
e^{i\theta} = \cos(\theta) + i\sin(\theta)
\end{equation}
\end{theorem}

As a special case, when $\theta = \pi$, we obtain Euler's identity:
\begin{equation}
e^{i\pi} + 1 = 0
\end{equation}

This remarkable equation connects five fundamental mathematical constants: $e$, $i$, $\pi$, $1$, and $0$.

\section{Differential Equations and Taylor Series}

Consider the differential equation:
\begin{equation}
\frac{d^2y}{dx^2} + p(x)\frac{dy}{dx} + q(x)y = f(x)
\end{equation}

The Taylor series expansion of a function $f(x)$ around point $a$ is:
\begin{equation}
f(x) = \sum_{n=0}^{\infty} \frac{f^{(n)}(a)}{n!}(x-a)^n = f(a) + f'(a)(x-a) + \frac{f''(a)}{2!}(x-a)^2 + \cdots
\end{equation}

\section{Advanced Calculus}

\subsection{Multiple Integration}
The triple integral over a region $E$ in $\mathbb{R}^3$ is:
\begin{equation}
\iiint_E f(x,y,z) \, dV = \int_{a}^{b} \int_{g_1(x)}^{g_2(x)} \int_{h_1(x,y)}^{h_2(x,y)} f(x,y,z) \, dz \, dy \, dx
\end{equation}

\subsection{The Fundamental Theorem of Calculus}
\begin{theorem}
If $f$ is continuous on $[a,b]$ and $F$ is an antiderivative of $f$, then:
\begin{equation}
\int_a^b f(x) \, dx = F(b) - F(a)
\end{equation}

Moreover, if $F(x) = \int_a^x f(t) \, dt$, then:
\begin{equation}
\frac{d}{dx}\int_a^x f(t) \, dt = f(x)
\end{equation}
\end{theorem}

\section{Linear Algebra and Matrix Operations}

For matrices $A \in \mathbb{R}^{m \times n}$ and $B \in \mathbb{R}^{n \times p}$, their product is:
\begin{equation}
(AB)_{ij} = \sum_{k=1}^{n} a_{ik}b_{kj}
\end{equation}

The determinant of a $2 \times 2$ matrix is:
\begin{equation}
\det\begin{pmatrix} a & b \\ c & d \end{pmatrix} = ad - bc
\end{equation}

\section{Number Theory and the Riemann Hypothesis}

The Riemann zeta function is defined for $\text{Re}(s) > 1$ by:
\begin{equation}
\zeta(s) = \sum_{n=1}^{\infty} \frac{1}{n^s} = \prod_{p \text{ prime}} \frac{1}{1-p^{-s}}
\end{equation}

The Riemann Hypothesis conjectures that all non-trivial zeros of $\zeta(s)$ have real part $\frac{1}{2}$.

\section{Probability and Statistics}

\subsection{The Normal Distribution}
The probability density function of the normal distribution is:
\begin{equation}
f(x) = \frac{1}{\sigma\sqrt{2\pi}} e^{-\frac{(x-\mu)^2}{2\sigma^2}}
\end{equation}

\subsection{Bayes' Theorem}
For events $A$ and $B$ with $P(B) \neq 0$:
\begin{equation}
P(A|B) = \frac{P(B|A)P(A)}{P(B)} = \frac{P(B|A)P(A)}{\sum_{i} P(B|A_i)P(A_i)}
\end{equation}

\section{Partial Differential Equations}

The heat equation in one dimension:
\begin{equation}
\frac{\partial u}{\partial t} = \alpha \frac{\partial^2 u}{\partial x^2}
\end{equation}

The wave equation in three dimensions:
\begin{equation}
\frac{\partial^2 u}{\partial t^2} = c^2 \nabla^2 u = c^2 \left(\frac{\partial^2 u}{\partial x^2} + \frac{\partial^2 u}{\partial y^2} + \frac{\partial^2 u}{\partial z^2}\right)
\end{equation}

\section{Complex Integrals and Residue Theory}

\begin{theorem}[Cauchy's Residue Theorem]
If $f$ is analytic inside and on a simple closed contour $C$ except at finitely many points $z_1, z_2, \ldots, z_n$, then:
\begin{equation}
\oint_C f(z) \, dz = 2\pi i \sum_{k=1}^{n} \text{Res}(f, z_k)
\end{equation}
\end{theorem}

\section{Functional Analysis}

For a Hilbert space $\mathcal{H}$ with inner product $\langle \cdot, \cdot \rangle$, the Cauchy-Schwarz inequality states:
\begin{equation}
|\langle x, y \rangle| \leq \|x\| \|y\|
\end{equation}

where $\|x\| = \sqrt{\langle x, x \rangle}$.

\section{Conclusion}

These mathematical expressions demonstrate various levels of complexity, from basic calculus to advanced topics in analysis and algebra. Each requires careful consideration for natural speech conversion.

\end{document}